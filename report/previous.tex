% S/C 

Jacques et al.\cite{sc} har tidligere vist at der kan opnås en massiv
ydelsesforbedring ved at parallelisere superposition/convolution og
lade grafikkortet om de hårde beregninger, så som raytracing til
bestemmelse af radiological depth.

% Siddon and fast raytracing

Til raytracing bliver der altid taget udgangspunkt i Siddon's linear
time raytracing algoritme fra 1985\cite{siddon}. Denne algoritme brød
med tidligere algoritmer, der alle opfattede CT data som en række
individuelle voxels, og i stedet anskuer den dataen som tre ortogonale
sæt af parallelle planer med samme afstand mellem sig.

Fox et al.\cite{fastraytracing} identificerede i 2005 problemet med at
beregne alle texture indices for alle stykker af strålen og viste
hvordan dette kunne gøres løbende mens strålen traces. De gjorde også
rede for en naiv implementation af raytracing, inspireret af
raycasting fra rendering af 3D legemer, ville lide under read-write
konflikter og hvordan dette problem kunne løses ved at skifte
synspunkt fra strålen til de enkelte voxels.
