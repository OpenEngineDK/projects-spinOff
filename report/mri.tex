\section{MRI}
\label{sec:MRI}

% What MRI does

Magnetic Resonance Imaging is a technique used for visualizing the
amount of hydrogen atoms present in a patient. Each voxel in the image
corresponds to a spin packet and the intensity in the voxel is an
approximation of how many hydrogen atoms was present within that
particular spin packet. This is useful in medical imaging as different
tissue will produce different intensities and can thus be distinguised
in the images.

% The B0 magnetic field

To control the direction of the hydrogen nucleus spin several
magnetic fields are employed. The first is a static field called
$\mathbf{B}_0$ aligned along the length of the patiens body from toe
to head. In clinical imaging this is taken as the z-axis. The effect
of applying $\mathbf{B}_0$ is to align all spin packets net
magnetization vectors along the z-axis. The net magnetization vector
is then said to be at equilibrium. This does not mean that all the
individual spins of the hydrogens nuclei are pointing upwards, but the
sum of all spindirections will produce a vector along the z-axis. The
individuel spins are also precessing about the z-axis, but while the
net magnetization is at equilibrium this precessing is not measurable.

% RF pulse

While at equilibrium the net magnetization vector does not produce any
signal. To induce a signal in the surrounding coils the net
magnetization will have to be flipped into the transverse plane,
ie. the xy-plane. To flip the magnetization another magnetic field,
$\mathbf{B}_1$, is used. This is called the RF field. Once the net
magnetization vector has been rotated into the transverse plane, the
RF field is turned off. The magnetic spin is again only affected by
$\mathbf{B}_0$ and will therefore return to equilibrium. Because the
net magnetization vector is still precessing about the z-axis, the
spin will then induce a signal in the surrounding coils, which can be
used to create the image.

% Distinquishing signals with Slice selection, frequency encoding and
% pulse encoding

If all net magnetizations are flipped at the same time it will be
impossible to distinguish their signal and the resulting image will
consist of one gray voxel. A gradient field consisting of several
smaller magnetic fields are therefore used to distinguish the
individuel spin packets.

The first is a slice selection gradient field, $\mathbf{B}_{G_s}$,
which when applied will make only one slice react to the
$\mathbf{B}_1$ field, and therefore the coil will only receive signals
from spin packets in that slice.

The second gradient field, $\mathbf{B}_{G_\phi}$, phase shifts the net
magnetizations transverse spin. What this means is that the transverse
spin of the net magnetizations are brought out of phase with the other
netmagnetizations in the slice, thus giving them a uniquely detectable
rotation. And the third gradient field, $\mathbf{B}_{G_f}$, is a
linear magnetic field gradient. The effect of $\mathbf{B}_{G_f}$ is to
relate the frequency at which the net magnetization precesses in the
transverse plane to its spatial location.

These last two gradients will make it possible for a fourier transform
to \emph{recognize} the individuel spin packets and recreate the original
image.

%%% Local Variables: 
%%% mode: latex
%%% TeX-master: "report.tex"
%%% TeX-PDF-mode: t
%%% End: 
