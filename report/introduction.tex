%% • The introduction must be dynamite. 
%% – The reader forms an oppinion of the work right from the start... 
%% • The introduction is an extension of the abstract. 
%% • Should be easy to read and understand 
%% • Should make it easy for anyone to tell
%% – What your paper is about 
%%   – What problem it solves
%%   – Why the problem and solution is interesting and relevant (motivation and context). Is it a long- outstanding problem?
%%   – What is new in your paper and how (much) does it improve on the strongest alternatives/previous work (include a few of the most relevant references here).

%% • Start the introduction with the motivation. Think in large contexts and don’t be afraid to be a poet.
%% • All implications, contributions and keypoints of your work must be included here.
%% • Make it very clear how your work will impact the future of medical image computing (will people use it?).
%% • If your work is pioneering, s-p-e-l-l i-t o-u-t.
%% • Briefly make it clear how you evaluate your method in the Results section.
%% • Make sure to explain where your method applies and where it does not apply (limitations).


%Vi arbejder med radiotherapi, stråleafsæt i patienter.
%BEDRE PERFORMANCE MED GPU, wuuuuuuau. Vi går efter realtime
%performance, kan rede liv.
%Flytte dele af beregningnen til gpu'en for at være fremtidssikret.
%Forskellige synspunkter ændre algoritmens karakter
%Superposition/convolution - afsæt terma og convolute/propagere det.
%Fast raytracing til radiologisk dybde
%Results section vil evaluere vores optimering af koden.

\IEEEPARstart{K}{ræft} behandling med intensity modulated radiation
therapy, IMRT, har længe været et tidskrævende problem indenfor
medicin. IMRT behandling er ansvarlige for at beregne strålekanonernes
doseafsætning i patienten, , og sørge for at minimerer denne stråling
og skaden strålingen kan foresage i patientens raske kropsdele, mens
strålingen maximeres hen over kræftsvulsten, så svulsten forsvinder
hurtigst muligt.

Fremgangsmåden for IMRT behandlinger i dag er at beregne patientens
behandling på dag et og så bruge denne beregning som udgangspunkt for
alle patientens videre behandlinger.

Selvom der bliver gjort meget for at placerer patienten i samme
stilling under hver behandling, er dette ikke præcist nok. En masse
faktore spiller nemlig ind omkring patientens krops placering; bl.a kan
patienten ændre kropsbygning ved at tabe sig eller patientens bryst
kan hæve og sænke sig, så svulsten i visse tilfælde bevæger sig
udenfor strålekanonernes påvirkning.

Det vil derfor være en fordel hvis det var muligt at foretage IMRT
beregningerne på patienten lige inden behandling påbegyndes, og gerne
løbende under behandling, for at mindske fejldosering i forhold til
ændringer i patientenskrop. Desværre kan det tage flere timer at
optimere en behandling, så dette er i øjeblikket ikke muligt.

Da processing units ikke længere bliver hurtigere, men i stedet får
flere kerner, betyder dette også at tidligere serielle algoritmer ikke
længere vil bliver hurtigere i takt med udviklingen og disse er derfor
nødt til at omskrives til parallelle algoritmer for at være
fremtidssikret.

I denne artikel fokuserer vi på at beregne dosen afsat i patienten ved
hjælp af Superposition/Convolution\cite{sc}. Vi vil gennem nVidias
CUDA framework aflaste CPU'en ved at flytte størstedelen af
beregningerne over på GPU'en og vise at med det rette synspunkt kan
raytracing udføres dataparallelt uden read-write konflikter. Et mål på
længere sigt vil være at kunne genberegne patientens IMRT behandling
for hver behandlingssession for at mindske fejl. Dette mål er en reel
mulighed med hurtigere hardware og algoritmer, der bedre kan udnytte
denne hardware.

% Forklaring af S/C
Som nævnt fokuserer vi i denne artikel på implementationen af
Superposition/Convolution, S/C. Vi vil hovedsageligt fokuserer på
raytracing og beregning af den radiologiske dybde i patienten, da
langt størstedelen, over 70\% \cite{fastraytracing}, af regnekraften
går til at løse disse problemer.

S/C beregner først hvor mange fotoner, der afsættes i patienten i et
bestemt område. Til dette kræves der viden om hvor meget materiele
strålen har rejst igennem. Her kommer den radiologiske dybde ind, som
er summen af densiteter vægtet med længden af strålens rejse gennem
disse.

De afsatte fotoner vil slå elektroner løs, og det er så disse
elektroner, der leverer den faktiske stråling og bekæmper
kraftcellerne. Den efterfølgende convolution del af algoritmen består
derfor i at finde ud af hvad disse elektroner påvirker og med hvor
meget. Til dette skal vi igen bruge den radiologiske afstand, da denne
afstand er med til at afgøre hvor mange af elektronerne vil påvirke et
område.

% Radiological depth via fast raytracing
Allerede nu skulle det være klart at vi ofte vil komme til at beregne
den radiologiske dybde og vi derfor er nødt til at overveje nogle
hurtige algoritmer til dette. Vi vil derfor tage udgangspunkt i Fox et
al. \cite{fastraytracing} og vise hvordan deres raytracing algoritme
kan optimeres ved at lave preberegniner inden strålen følges og
argumenterer for den geometriske forståelse af vores
optimeringer/preberegninger.

% Vores resultater (Ingen cpu impl, men fokus på optimering af algoritmer)
Vores resultater vil ikke være bundet i nye algoritmer til medicinsk
billedbehandling eller skiftet fra serielle til parallelle algoritmer,
dette behandles allerede i masser af artikler i forvejen og bør være
slået fast. Vi vil i stedet fokuserer på optimering af algoritmerne,
som giver mening i forhold til CUDA, og i nogen grad den performance
forbedring disse giver.
