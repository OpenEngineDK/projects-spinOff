\section{Results and Conclusion}

The main objectives of this project was to improve our understanding
of MRI physics. In this regard we are succesful. Our secondary
objective however did not completely succeed since we were unable to
produce a meaningful output based on the input image. We can identify
several causes of this problem. First of all our construction of the
initial spin parameters based on an image might not be valid at
all. We should eliminate this error by simply using valid input data.
Furthermore we see another problem in the sampling to K-space. In our
implementation, the sampling is done from the lower left corner in
K-space, this will make us loose some resolution. Instead the data
should be centered in the middle of the K-space. The timing and field
strength of the gradients are at the moment most likely not correct. 

On the positive side our we get some good visualizations of the spins
and are able to observe changes in spin relaxations based on $T1$ and
$T2$ values. These behavours seem to be correct in the sense that
greater values yield a longer relaxation time. We also observe changes
in the acquired signal based on both the RF and the gradient fields.

Alltogether we have achieved a good foundation for further development
of a working simulator. We believe that our problems with the gradient
fields can be solved by finding correct parameters and by focusing on
correct timing of the FID sequences.



%%% Local Variables: 
%%% mode: latex
%%% TeX-master: "report.tex"
%%% TeX-PDF-mode: t
%%% End: 
