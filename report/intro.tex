%% • The introduction must be dynamite. 
%% – The reader forms an oppinion of the work right from the start... 
%% • The introduction is an extension of the abstract. 
%% • Should be easy to read and understand 
%% • Should make it easy for anyone to tell
%% – What your paper is about 
%%   – What problem it solves
%%   – Why the problem and solution is interesting and relevant (motivation and context). Is it a long- outstanding problem?
%%   – What is new in your paper and how (much) does it improve on the strongest alternatives/previous work (include a few of the most relevant references here).

%% • Start the introduction with the motivation. Think in large contexts and don’t be afraid to be a poet.
%% • All implications, contributions and keypoints of your work must be included here.
%% • Make it very clear how your work will impact the future of medical image computing (will people use it?).
%% • If your work is pioneering, s-p-e-l-l i-t o-u-t.
%% • Briefly make it clear how you evaluate your method in the Results section.
%% • Make sure to explain where your method applies and where it does not apply (limitations).


\section{Introduction}

% MRI, the magnetic fields, imaging

\IEEEPARstart{M}{edical} resonance imaging, MRI, is a technique used
mostly in clinical imaging to produce 3D images of patients. The
technique relies on several magnetic fields to control and measure the
spin property of atoms in the human body. Measuring every single atom
is an infeasable task and the atoms are therefore grouped into
discrete spin packets, whose net magnetization vector is a
representation of the collective spin of all atoms in the spin
packet. This net magnetization is then measured through an electric
signal it induces in the surrounding coils, postprocessed and fourier
transformed from a k-space image and back into image space.

% Focus

In this report we will focus on the basic physics of MRI and use it to
simulate the process of image acquisition. To keep the simulation
simple our program will not simulate the positioning of the net
magnetization in preparation for the signal acquisition. Here we will
instead assume that the theory is correct and analytically position
the net magnetization vectors. We will then proceed with the
simulation of the signal acquisition and storing the signal in
k-space.

% Why GPGPU?

Due to both the parallel nature of the spin packets affected by the
magnetic fields and the signal acquisition being performed identically
at discrete time intervals, MRI lends itself perfectly to dataparallel
programming. For this reason our simulator will be developed in
NVIDIA's \textbf{Compute Unified Device Architecture}, CUDA, and we
will explorer how best to utilise the Graphics Programming Unit, GPU.

% TODO Reults
TODO results

%%% Local Variables:
%%% mode: latex
%%% TeX-master: t
%%% TeX-PDF-mode: t
%%% End:
