\section{Implementation}

% model the B0 and BGf fields

% Assume slice selection and possibly phase gradient

In our simulation we assume that the correct slice has been selected
and excited, so we only work with a 2 dimensional image and the net
magnetization vectors have been rotated into the transversal
plane. This allows us to focus on the image acquisition phase.

% Bloch equations with the gradient and B0, which is what we're
% solving

With only the static magnetic field $\mathbf{B}_0$ and
$\mathbf{B}_{G_f} = \langle 0, 0, p_x * B_{G_f} \rangle$ active, the
block equations simply to

\begin{displaymath}
  \begin{array}{l}
    \frac{dM_x}{dt} = \gamma (B_0 + p_x * B_{G_f}) M_y - \frac{M_x}{T_2} \\
    \frac{dM_y}{dt} = - \gamma (B_0 + p_x * B_{G_f}) M_x - \frac{M_y}{T_2} \\
    \frac{dM_z}{dt} = - \frac{M_z - M_{eq}}{T_1}
  \end{array}
\end{displaymath}

which have the analytical solution 

\begin{displaymath}
  \begin{array}{l}
    M_x(t) = e^{-t/T_2}(M_x(0) \cos(w_0 t + w_f t) - M_y(0) \sin (w_0 t + w_f t)) \\
    M_y(t) = e^{-t/T_2}(M_x(0) \sin(w_0 t + w_f t) + M_y(0) \cos (w_0 t + w_f t)) \\
    M_z(t) = M_z(0) e^{t/T_1} + M_{eq}(1 - e^{-1/T_1})
  \end{array}
\end{displaymath}

% One kernel for Calculating several/all timesteps and store them in
% an array. (one thread pr voxel in the original image)

% Then have one kernel sum up the signal and write it in out k-space
% image.

\subsection{Sampling}

When sampling the signal, we assume that we have a perfect receiving
coil. One where the output signal is proportional the same as $M_x$
and $M_y$.

When the kernel is done, we have a array containing $M$ for each spin
packet. To generate the signal, we simply summarize all the packet
into one vector ($s$). $s_x$ and $s_y$ represents the signal for a
single sample.

In our implementation, the sampling is synchronized to our time step,
so each time we generate a new $M$, we store a sample of the signal.

After recording one line in our sampling space, we relax the spins and
add a phase encoding gradient. Then the recording process starts over
on the next line.

Our sample space is then copied into K-space, where we use the
inverse discrete fourier transformation to transform it into image-space.

\subsection{Problems}

One of our problems is in the sampling to K-space. In our
implementation, the sampling is done from the lower left corner in
K-space, this will make us loose some resolution. Instead the data
should be centered in the middle of the K-space.

% A last kernel will transform the image back to image space. Use
% Sangilds fft wrapper.

% If this image looks like crap then we don't care! Some mumbo jumbo
% about where it might have gone wrong, possible some images showing
% the net magnetization (with color interpretation) and then that's
% all she wrote.

%%% Local Variables: 
%%% mode: latex
%%% TeX-master: "report.tex"
%%% TeX-PDF-mode: t
%%% End: 
