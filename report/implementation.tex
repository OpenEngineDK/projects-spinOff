\section{Implementation}

% model the B0 and BGf fields

% Assume slice selection and possibly phase gradient

In our simulation we assume that the correct slice has been selected
and excited, so we only work with a single two-dimensional image. This
allows us to focus on the image acquisition phase while applying the
phase- and frequency gradients, respectively.

The core of the Bloch equation simulation is implemented in an
iterative cuda kernel. The most interesting code lines are shown in
figure \ref{fig:kernel}.

% \begin{strip}
\begin{figure*}
  \centering
\begin{lstlisting}
    // fid rotation
    m = rotX(b.x)*m;
    // relaxation
    m += make_float3(-m.x*dtt2, -m.y*dtt2, (props[idx].eq-m.z)*dtt1);
    // gradient rotation
    m = rotZ(GYROMAGNETIC_RATIO*gx*posX*dt)*m;
    m = rotZ(GYROMAGNETIC_RATIO*gy*posY*dt)*m;
    // reference position
    ref_spins[idx] = m;
    // reference to laboratory
    lab_spins[idx] = make_float3(m.x * cos(omega * thetime) - m.y * sin(omega*thetime), 
                     m.x * sin(omega * thetime) + m.y * cos(omega*thetime),  
                     m.z);
\end{lstlisting}
  \caption{A code snippet of the simulation kernel.}
  \label{fig:kernel}
\end{figure*}
% \end{strip}

Our kernel keeps track of the spin packets in two different coordinate
systems, namely the rotating frame of reference and the laboratory
frame. As previously mentioned, the rotating frame of reference,
rotates at a frequency of $\omega_0=\gamma B_0$. From the perspective
of this coordinate system we see no rotation about the z-axis, when
$B_0$ is the only affecting field. This makes it easier to simulate
the effects of the RF field and the gradient fields. As can be seen our
kernel is divided into four different steps. 

We first simulate the effect of the RF field by simply rotating each
spin vector about the x-axis, at a specified angle. For our purposes
this is always a 90 degree angle to rotate the spin vector onto the
transversal plane. This step makes sense since we operate in the
rotating frame of reference.

The second step uses the individual $T1$ and $T2$ relaxation times to
bring the spin vector back into its original equilibrium state. This
means that transversal components are gradually zeroed out while the
z-component gradually regains its initial magnitude.

In the third step we apply the effect of the gradient fields. Both
fields simply induce a rotation around the z-axis by an angle which is
dependent on the field strength and the position of the spins. This
makes each spin rotate either faster or slower than $\omega_0$ and is
seen in the rotating frame of reference as either a positive or
negative rotation. 

In the last step we compute the orientation of the spin vectors in the
laboratory frame of reference. This is the actual global space
orientation of the spins containing the $\omega_0$ rotation together
with the angular offsets resulting from the gradient fields. We use
these vectors later on in our data acquisition phase.

\subsection{Simulating an FID sequence}

The cuda kernel as described in the previous section drives the entire
simulation in fixed size timestep manner. But in order to achieve a
meaningful sequence of spins we need to carefully coordinate the
parameters sent to the kernel in each step. An FID sequence assumes
that the spins are in their equilibrium state and applies the following steps
\begin{itemize}
\item Apply 90 degree RF signal.
\item Apply phase gradient for the duration of $\tau$ seconds.
\item Apply frequency gradient and start sampling of spin data.
\item Wait until the spins reach equilibrium and repeat sequence with
  a slightly different phase gradient field strength.
\end{itemize}
We simulate the RF signal step by simply calling the kernel with a
timestep of $0$ seconds and setting the rotation angle to
$\frac{\pi}{2}$. This flips the vectors without the trouble of
simulating how the RF field affects the spins over time. This is fine
since we are only interested in the resulting effect which is the 90
degree rotation. Next we do a series of kernel iterations with a
specific phase gradient field strength applied. This value varies
linearly for each FID sequence simulation. This has the effect of
changing the frequency of each spin based on their spatial
y-coordinate. We now do a series of kernel iterations with the phase
gradient turned of and the frequency gradient set at a constant field
strength. Since the phase gradient is turned off the spins in the
y-direction now rotates at the same frequency again but with a
different phase. The frequency gradient makes the spins along the
x-direction vary in frequency and we can now start sampling the signal
between each kernel iteration.

% Bloch equations with the gradient and B0, which is what we're
% solving

% With only the static magnetic field $\mathbf{B}_0$ and
% $\mathbf{B}_{G_f} = \langle 0, 0, p_x * B_{G_f} \rangle$ active, the
% block equations simply to

% \begin{displaymath}
%   \begin{array}{l}
%     \frac{dM_x}{dt} = \gamma (B_0 + p_x * B_{G_f}) M_y - \frac{M_x}{T_2} \\
%     \frac{dM_y}{dt} = - \gamma (B_0 + p_x * B_{G_f}) M_x - \frac{M_y}{T_2} \\
%     \frac{dM_z}{dt} = - \frac{M_z - M_{eq}}{T_1}
%   \end{array}
% \end{displaymath}

% which have the analytical solution 

% \begin{displaymath}
%   \begin{array}{l}
%     M_x(t) = e^{-t/T_2}(M_x(0) \cos(w_0 t + w_f t) - M_y(0) \sin (w_0 t + w_f t)) \\
%     M_y(t) = e^{-t/T_2}(M_x(0) \sin(w_0 t + w_f t) + M_y(0) \cos (w_0 t + w_f t)) \\
%     M_z(t) = M_z(0) e^{t/T_1} + M_{eq}(1 - e^{-1/T_1})
%   \end{array}
% \end{displaymath}

% One kernel for Calculating several/all timesteps and store them in
% an array. (one thread pr voxel in the original image)

% Then have one kernel sum up the signal and write it in out k-space
% image.

\subsection{Sampling}

When sampling the signal, we assume that we have a perfect receiving
coil. One where the output signal is proportional to $M_x$ and $M_y$.

When the kernel is done, we have a array containing $\textbf{M}$ for each spin
packet. To generate the signal, we simply summarize all the packet
into one vector ($s$). $s_x$ and $s_y$ represents the signal for a
single sample.

In fig. \ref{fig:reduce} is a CUDA kernel that summarizes the data into a array with one
entry for each block. This array is then summarized on the CPU.

\begin{figure*}[h]
\begin{lstlisting}
template <class T>
__global__ void
reduce3(T *g_idata, T *g_odata, unsigned int n, T zero)
{
    T *sdata = SharedMemory<T>();

    // perform first level of reduction,
    // reading from global memory, writing to shared memory
    unsigned int tid = threadIdx.x;
    unsigned int i = blockIdx.x*(blockDim.x*2) + threadIdx.x;

    T mySum = (i < n) ? g_idata[i] : zero;
    if (i + blockDim.x < n) 
        mySum += g_idata[i+blockDim.x];  

    sdata[tid] = mySum;
    __syncthreads();

    // do reduction in shared mem
    for(unsigned int s=blockDim.x/2; s>0; s>>=1) 
    {
        if (tid < s) 
        {
            sdata[tid] = mySum = mySum + sdata[tid + s];
        }
        __syncthreads();
    }

    // write result for this block to global mem 
    if (tid == 0) g_odata[blockIdx.x] = sdata[0];
}
\end{lstlisting}
  \caption{Summarization}
  \label{fig:reduce}
\end{figure*}

In our implementation, the sampling is synchronized to our time step,
so each time we generate a new $M$, we store a sample of the signal.

After recording one line in our sampling space, we relax the spins and
add a phase encoding gradient. Then the recording process starts over
on the next line.

Our sample space is then copied into K-space, where we use the inverse
discrete fourier transformation (IDFT) to transform it into
image-space.


% A last kernel will transform the image back to image space. Use
% Sangilds fft wrapper.

% If this image looks like crap then we don't care! Some mumbo jumbo
% about where it might have gone wrong, possible some images showing
% the net magnetization (with color interpretation) and then that's
% all she wrote.

%%% Local Variables: 
%%% mode: latex
%%% TeX-master: "report.tex"
%%% TeX-PDF-mode: t
%%% End: 
