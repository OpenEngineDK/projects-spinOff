\section{Future Work}

%%% Local Variables: 
%%% mode: latex
%%% TeX-master: "report.tex"
%%% TeX-PDF-mode: t
%%% End: 

One of the biggest problems in our current implementation is the
timing issue. Right now almost every time related variable is a user
defined value. We need to investigate what would be good values for the
following variables based on their relation with the rest of the simulation
framework:
\begin{itemize}
\item The size of the time steps.
\item The duration of the phase gradient field application.
\item The sampling rate.
\end{itemize}
Furthermore we need to fully understand how the sampled data should be
handled by the fourier transform in order to reconstruct the image slice.

When the above problems are solved the next step will be to extend our
simulatior with more advanced features, such as $B_0$ field
inhomogenities, changes in magnetic fields due to chemical reactions
in tissue, and handling of three-dimensional data sets.

Our current work forms a good basis for exploring these aspects of MRI
and due to the parallel implementation of the simulation kernel we can
also hope to achieve a very efficient simulator.
