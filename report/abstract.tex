%% Isbrief:200-300words. 
%% • Summarizes your work:
%% – What problem does this paper solve?
%% – Why is this problem important for medical image computing (the context and motivation)? 
%% – How does your method work? 
%% – How does it differ from previous work? 
%% – How much better than previous work is it?
%% • There are usually no references in the abstract
%% • The abstract should be readable and understandable by non-experts

%Paralleliser gøjlet for øget hastighed. (lær at gøre det)
%Dose beregning er langsomt og ikke interaktivt endnu, parallelisering
%kan afhjælpe dette.
%Tager udgangspunkt i superposition/convolution, hvor radiological
%depth bliver beregnet vha fast raytracing.
%Gør det ikke. Men måske mere dybdegående forklaring af vores
%algoritme. (Fabs udenfor loop I hope)


Algoritmer til medicinsk billedbehandling har længe nydt godt af
Moore's lov, at cpu hastigheder blev fordoblet hver 18 måned. Dette er
ikke tilfældet længere, hvor cpuer i stedet fordobler antallet af
kerner, og algoritmerne må derfor følge med og omskrives til let
paralleliserbare algoritmer.

I denne artikel sætter vi fokus på radioterapi, nærmere bestemt
afsættelse af radioaktiv stråling i patienter og hvordan denne spreder
sig gennem patienten. Med baggrund i nVidias \textbf{Compute Unified
  Device Architecture}, CUDA, vil vi argumenterer for brugen af
multikerne gpu'er til at løse dette problem og hvordan det er muligt
at dataparalleliserer stråleafsætningen samt dens spredning.

Vores implementation vil være todelt; først vil vi implementere
uoptimerede CUDA kerner, hvor teorien vil være let at genkende i
programmets algoritmer. Dernæst vil vi diskuterer og implementere
forskellige optimeringer.
