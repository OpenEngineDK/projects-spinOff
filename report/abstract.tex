%% Isbrief:200-300words. 
%% • Summarizes your work:
%% – What problem does this paper solve?
%% – Why is this problem important for medical image computing (the context and motivation)? 
%% – How does your method work? 
%% – How does it differ from previous work? 
%% – How much better than previous work is it?
%% • There are usually no references in the abstract
%% • The abstract should be readable and understandable by non-experts

%Paralleliser gøjlet for øget hastighed. (lær at gøre det)
%Dose beregning er langsomt og ikke interaktivt endnu, parallelisering
%kan afhjælpe dette.
%Tager udgangspunkt i superposition/convolution, hvor radiological
%depth bliver beregnet vha fast raytracing.
%Gør det ikke. Men måske mere dybdegående forklaring af vores
%algoritme. (Fabs udenfor loop I hope)

In this report we present the basic principles of the physics behind
magnetic resonance imaging (MRI). Based on these we present our work
on a parallel CUDA-based implementation of an MRI-simulator. The main
kernel of the simulator is based on a discrete time-step solution to
the Bloch equations. The simulator is still at a very early stage of
development and as such does not produce correct final results. It
does however handle the effects of simple RF signals together with the
effect of phase and frequency gradients. Furthermore it samples and
visualizes the resulting signal.

%%% Local Variables: 
%%% mode: latex
%%% TeX-master: "report.tex"
%%% TeX-PDF-mode: t
%%% End: 
